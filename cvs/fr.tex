\documentclass[letterpaper,11pt]{article}

\usepackage{latexsym}
\usepackage[empty]{fullpage}
\usepackage{titlesec}
\usepackage{marvosym}
\usepackage[usenames,dvipsnames]{color}
\usepackage{verbatim}
\usepackage{enumitem}
\usepackage[hidelinks]{hyperref}
\usepackage{fancyhdr}
\usepackage[french]{babel}
\usepackage{tabularx}
\input{glyphtounicode}


%----------FONT OPTIONS----------
% sans-serif
% \usepackage[sfdefault]{FiraSans}
% \usepackage[sfdefault]{roboto}
% \usepackage[sfdefault]{noto-sans}
% \usepackage[default]{sourcesanspro}

% serif
% \usepackage{CormorantGaramond}
% \usepackage{charter}


\pagestyle{fancy}
\fancyhf{} % clear all header and footer fields
\fancyfoot{}
\renewcommand{\headrulewidth}{0pt}
\renewcommand{\footrulewidth}{0pt}

% Adjust margins
\addtolength{\oddsidemargin}{-0.5in}
\addtolength{\evensidemargin}{-0.5in}
\addtolength{\textwidth}{1in}
\addtolength{\topmargin}{-.5in}
\addtolength{\textheight}{1.0in}

\urlstyle{same}

\raggedbottom
\raggedright
\setlength{\tabcolsep}{0in}

% Sections formatting
\titleformat{\section}{
  \vspace{-4pt}\scshape\raggedright\large
}{}{0em}{}[\color{black}\titlerule \vspace{-5pt}]

% Ensure that generate pdf is machine readable/ATS parsable
\pdfgentounicode=1

%-------------------------
% Custom commands
\newcommand{\resumeItem}[1]{
  \item\small{
    {#1 \vspace{-2pt}}
  }
}

\newcommand{\resumeSubheading}[4]{
  \vspace{-2pt}\item
    \begin{tabular*}{0.97\textwidth}[t]{l@{\extracolsep{\fill}}r}
      \textbf{#1} & #2 \\
      \textit{\small#3} & \textit{\small #4} \\
    \end{tabular*}\vspace{-7pt}
}

\newcommand{\resumeSubSubheading}[2]{
    \item
    \begin{tabular*}{0.97\textwidth}{l@{\extracolsep{\fill}}r}
      \textit{\small#1} & \textit{\small #2} \\
    \end{tabular*}\vspace{-7pt}
}

\newcommand{\resumeProjectHeading}[2]{
    \item
    \begin{tabular*}{0.97\textwidth}{l@{\extracolsep{\fill}}r}
      \small#1 & #2 \\
    \end{tabular*}\vspace{-7pt}
}

\newcommand{\resumeSubItem}[1]{\resumeItem{#1}\vspace{-4pt}}

\renewcommand\labelitemii{$\vcenter{\hbox{\tiny$\bullet$}}$}

\newcommand{\resumeSubHeadingListStart}{\begin{itemize}[leftmargin=0.15in, label={}]}
\newcommand{\resumeSubHeadingListEnd}{\end{itemize}}
\newcommand{\resumeItemListStart}{\begin{itemize}}
\newcommand{\resumeItemListEnd}{\end{itemize}\vspace{-5pt}}

%-------------------------------------------
%%%%%%  CV  %%%%%%%%%%%%%%%%%%%%%%%%%%%%


\begin{document}

\begin{center}
    \textbf{\Huge \scshape Lucas Duport}
\end{center}

\begin{center}
\small +33 6 87 61 33 64 $|$
\href{mailto:cv.lucasduport@icloud.com}{\underline{cv.lucasduport@icloud.com}} $|$
\href{https://www.lucasduport.cc}{\underline{lucasduport.cc}}
\end{center}

%-----------PROFIL-----------
\section*{Profil}
\noindent
Étudiant en dernière année à l’EPITA, spécialisé en Intelligence Artificielle et Vision par Ordinateur. Je suis à la recherche d’un stage de fin d’études de 6 mois à partir de février 2026 pour mettre en pratique mes compétences et contribuer à des projets à fort impact au sein de votre organisation.

%-----------FORMATION-----------
\section{Formation}
  \resumeSubHeadingListStart
    \resumeSubheading
      {EPITA - Diplôme d’ingénieur en informatique}{Paris, France}
      {Spécialisation : Intelligence Artificielle \& Vision par Ordinateur | Membre d’une association étudiante}{2021 -- 2026}
\vspace{0.1em}
\resumeSubheading
  {San Francisco State University - Informatique}{San Francisco, CA}
  {Semestre d’échange — \textit{Ambassadeur EPITA}}{2023}
  \resumeSubHeadingListEnd


%-----------EXPÉRIENCE PROFESSIONNELLE-----------
\section{Expérience Professionnelle}
  \resumeSubHeadingListStart

    \resumeSubheading
      {STMicroelectronics}{Grenoble, France}
      {Stagiaire Data Scientist}{2025 (6 mois)}
      \resumeItemListStart
        \resumeItem{Développement d’un pipeline d’augmentation de données avec \textbf{RAG} pour la synthèse de textes techniques.}
        \resumeItem{Création d’une plateforme full-stack avec FastAPI et une interface en HTML/CSS/JavaScript pour la validation et l’annotation de données.}
        \resumeItem{Conception et amélioration d’un \textbf{système de suggestion de causes} en fonction de symptômes en collaboration avec les utilisateurs finaux.}
      \resumeItemListEnd

    \resumeSubheading
      {Kerquest SAS}{Paris, France}
      {Stagiaire Développement Logiciel}{2023 (2 mois)}
      \resumeItemListStart
          \resumeItem{Développement d’un prototype de système de vision par ordinateur pour \textbf{détecter des objets sur une grille} dans une ligne de production.}
        \resumeItem{Création d’une interface opérateur pour l’assistance à la détection en temps réel.}
        \resumeItem{Automatisation de l’installation de serveurs avec Debian Preseed, Ansible et des scripts Bash.}
      \resumeItemListEnd

    \resumeSubheading
      {EPITA}{Paris, France}
      {Assistant Professeur}{2023 -- 2026}
      \resumeItemListStart
        \resumeItem{Membre d’une équipe de 60 étudiants encadrant plus de 800 élèves de 3\textsuperscript{e} année dans leurs activités de programmation.}
        \resumeItem{Assistance au \textbf{debugging et réponses aux questions techniques} dans plusieurs langages.}
        \resumeItem{Conception de travaux pratiques et déploiement de systèmes de correction automatique pour plus de 600 étudiants de 1\textsuperscript{re} année.}
      \resumeItemListEnd

  \resumeSubHeadingListEnd

%-----------PROJETS-----------
\section{Projets}
  \resumeSubHeadingListStart

\resumeSubheading
  {Analyse d’avis TripAdvisor}{Groupe de 5}
  {Python}{2025}
  \resumeItemListStart
    \resumeItem{Développement d’un pipeline pour \textbf{classer les avis d’hôtels et générer} du texte synthétique à partir de mots-clés et notes.}
    \resumeItem{Implémentation et comparaison de régression logistique, RNNs, FNNs et transformers.}
    \resumeItem{Traitement du déséquilibre des classes avec \textbf{nlpaug}, augmentant fortement la précision.}
  \resumeItemListEnd

\resumeSubheading
  {Lecteur \& Solveur de Sudoku}{Groupe de 4}
  {C}{2023}
  \resumeItemListStart
    \resumeItem{Développement d’un système complet pour \textbf{détecter, extraire et résoudre} des Sudoku à partir d’images, intégralement en C, sans bibliothèque externe.}
    \resumeItem{Application de prétraitements : conversion en niveaux de gris, filtres de Gauss et de Sobel, redimensionnement manuel.}
    \resumeItem{Intégration d’un classificateur de chiffres et d’un solveur personnalisé dans une interface utilisateur intuitive.}
  \resumeItemListEnd

  \resumeSubHeadingListEnd
  
%-----------COMPÉTENCES & INTÉRÊTS-----------
\section{Compétences \& Centres d’intérêt}
\begin{itemize}[leftmargin=0.15in, label={}]
  \small{\item{
    \textbf{Langages}{ : Python, C, C++, Java, SQL, JavaScript, Bash, C\#} \\
    \textbf{Outils}{ : Git/GitHub/GitLab, Docker, Linux.} \\
    \textbf{Langues}{ : Français (natif), Anglais (\textbf{970/990 TOEIC}), Italien (notions).} \\
    \textbf{Savoir-être}{ : Curieux, passionné, bon communicant, calme, positif, travailleur.} \\
    \textbf{Centres d’intérêt}{ : Homelab, ski, horlogerie, poker, voyages.}
  }}
\end{itemize}

\end{document}
